
\chapter*{Abstract}

\addcontentsline{toc}{section}{Abstract}

\blindtext

\selectlanguage{German}%

\chapter*{Zusammenfassung}

\addcontentsline{toc}{section}{Zusammenfassung}

Dies hier ist ein Blindtext zum Testen von Textausgaben. Wer diesen Text liest, ist
selbst schuld. Der Text gibt lediglich den Grauwert der Schrift an. Ist das wirklich so?
Ist es gleichg�ltig, ob ich schreibe: ?Dies ist ein Blindtext? oder ?Huardest gefburn??
Kjift ? mitnichten! Ein Blindtext bietet mir wichtige Informationen. An ihm messe ich
die Lesbarkeit einer Schrift, ihre Anmutung, wie harmonisch die Figuren zueinander
stehen und pr�fe, wie breit oder schmal sie l�uft. Ein Blindtext sollte m�glichst viele
verschiedene Buchstaben enthalten und in der Originalsprache gesetzt sein. Er mu�
keinen Sinn ergeben, sollte aber lesbar sein. Fremdsprachige Texte wie ?Lorem ipsum?
dienen nicht dem eigentlichen Zweck, da sie eine falsche Anmutung vermitteln.

\selectlanguage{english}

\chapter*{Preface}

\addcontentsline{toc}{section}{Preface}

This thesis is submitted as partial fulfillment for obtaining the degree of Doctor of Philosophy
(PhD) at the Technical University of Munich (TUM). The PhD project
was financed by the ...

\vspace{\fill}


\begin{center}
Munich, INSERT DATE
\par\end{center}

\vspace{0.1cm}


\begin{center}
INSERT SIGNATURE
\par\end{center}

\begin{center}
Firstname Lastname
\par\end{center}
